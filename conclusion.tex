%%%%%%%%%%%%%%%%%%%%%%%%%%%%%%%%%%%%%%%%%%%%%%%%%%%%%%%%%%%%%%%%%%%%%%%%%%%%%%%%
\conclusion
%%%%%%%%%%%%%%%%%%%%%%%%%%%%%%%%%%%%%%%%%%%%%%%%%%%%%%%%%%%%%%%%%%%%%%%%%%%%%%%%

Целью данной работы является разработка метода обнаружения клонов, использующего нейронные сети. В работе кратко рассмотрена предметная область, введены основные определения и классификации, выделены главные требования к предлагаемому методу. В рамках исследования были рассмотрены существующие подходы к обнаружению клонов. Среди них были выделены наиболее подходящие для использования в предлагаемом методе, а именно - позволяющие осуществлять поиск клонов I-III типов.

Основная идея предлагаемого метода заключается в использовании нейронных сетей на стадии поиска клонов, а не на стадии предобработки. Данный подход позволяет проще находить клоны первых трех типов. В данной работе представлено полное описание предлагаемого метода, включая все необходимые и вспомогательные алгоритмы. 

На базе предложенного метода был разработан прототип инструмента обнаружения клонов. Также было проведено тестирование разработанного прототипа на выборках различного размера из набора данных BigCloneBench. Тестирование показало целесообразность использования данного метода в задачах поиска клонов. Однако, без должной оптимизации, время вычисления будет расти пропорционально размеру анализируемого проекта.

Дальнейшее развитие работы может выполняться в нескольких направлениях:

\begin{itemize}
\setlength\itemsep{0mm}
\item Оптимизация работы прототипа;
\item Реализация возможности поиска клонов IV типа.
\end{itemize}
