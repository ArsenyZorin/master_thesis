\chapter{Тестирование и анализ результатов}

В данном разделе проводится исследование характеристик прототипа на примере его использования для обнаружения клонов. Тестирование проводилось на нескольких различных выборках из BigCloneBench.

\section{Описание тестовых данных}

Для тестирования разработанного прототипа из выборки BigCloneBench были выделены партии с разным количеством методов:

\begin{itemize}
\setlength\itemsep{0mm}
\item приблизительно 4000 методов 
\item приблизительно 8000 методов
\item приблизительно 10000 методов
\item приблизительно 20000 методов
\end{itemize}

Выбранные фрагменты набора можно считать импровизированными <<проектами>> среднего, так и программами большого размера. Так как в выборке BigCloneBench присутствуют различные файлы, классы и методы из различных проектов, то случайно выбранные методы, вероятнее всего, друг с другом никак не связаны.

Как было описано ранее, в выборке данных BigCloneBench содержатся клоны всех типов (I-IV). BigCloneBench представляет из себя экспортированную базу данных формата PostgresSQL, состоящую из нескольких таблиц. Клоны и методы, которые не являются клонами в ней представлены как записи пар индексов всех методов с дополнительными параметрами. Примеры клонов трех типов представлены на рис.~\ref{fig:first}-\ref{fig:third}.

Преимуществом выборки BigCloneBench является ее заполнение с помощью ручного поиска клонов. Для этого был использован большой межпроектный репозиторий IJaDataset 2.0 (25 тыс. подсистем, 365 MLOC). Текущая версия данной выборки содержит в себе приблизительно 6.3 млн пар клонов и 262.5 тыс. не клонов~\cite{bcb}.

\begin{figure}[H]
\begin{subfigure}{\textwidth}
\begin{lstlisting}[style=java]
public static void copyFile(File src, File dst) throws IOException {
    InputStream in = new FileInputStream(src);
    OutputStream out = new FileOutputStream(dst);
    byte[] buf = new byte[1024];
    int len;
    while ((len = in.read(buf)) > 0) out.write(buf, 0, len);
    in.close();
    out.close();
}
\end{lstlisting}
\caption{Метод из файла selected/77754.java}
\end{subfigure}
\begin{subfigure}{\textwidth}
\begin{lstlisting}[style=java]
public static void copyFile(File src, File dst) throws IOException {
    InputStream in = new FileInputStream(src);
    OutputStream out = new FileOutputStream(dst);
    byte[] buf = new byte[1024];
    int len;
    while ((len = in.read(buf)) > 0) out.write(buf, 0, len);
    in.close();
    out.close();
}
\end{lstlisting}
\caption{Метод из файла selected/2077739.java}
\end{subfigure}
\caption{Пример клона I типа из BigCloneBench}
\label{fig:first}
\end{figure}

\begin{figure}[H]
\begin{subfigure}{\textwidth}
\begin{lstlisting}[style=java]
public static void copyFile(File source, File dest) throws IOException {
    FileChannel in = null, out = null;
    try {
        in = new FileInputStream(source).getChannel();
        out = new FileOutputStream(dest).getChannel();
        long size = in.size();
        MappedByteBuffer buf = in.map(FileChannel.MapMode.READ_ONLY, 0, size);
        out.write(buf);
    } finally {
        if (in != null) in.close();
        if (out != null) out.close();
    }
}
\end{lstlisting}
\caption{Метод из файла selected/166275.java}
\end{subfigure}
\end{figure}
\begin{figure}[H]\ContinuedFloat
\begin{subfigure}{\textwidth}
\begin{lstlisting}[style=java]
public static void fileCopy(File source, File dest) throws IOException {
    FileChannel in = null, out = null;
    try {
        in = new FileInputStream(source).getChannel();
        out = new FileOutputStream(dest).getChannel();
        long size = in.size();
        MappedByteBuffer buf = in.map(FileChannel.MapMode.READ_ONLY, 0, size);
        out.write(buf);
    } finally {
        if (in != null) in.close();
        if (out != null) out.close();
    }
}
\end{lstlisting}
\caption{Метод из файла selected/2558574.java}
\end{subfigure}
\caption{Пример клона II типа из BigCloneBench}
\label{fig:second}
\end{figure}

Задача предлагаемого метода заключается в нахождении клонов первых трех типов (I-III). Основываясь на этой задаче, необходимо произвести фильтрацию значений из базы данных. 

\begin{figure}[H]
\begin{subfigure}{\textwidth}
\begin{lstlisting}[style=java]
public void downloadFile(OutputStream os, int fileId) throws IOException, SQLException {
    Connection conn = null;
    try {
        conn = ds.getConnection();
        Guard.checkConnectionNotNull(conn);
        PreparedStatement ps = conn.prepareStatement("select * from FILE_BODIES where file_id=?");
        ps.setInt(1, fileId);
        ResultSet rs = ps.executeQuery();
        if (!rs.next()) {
            throw new FileNotFoundException("File with id=" + fileId + " not found!");
        }
        Blob blob = rs.getBlob("data");
        InputStream is = blob.getBinaryStream();
        IOUtils.copyLarge(is, os);
    } finally {
        JdbcDaoHelper.safeClose(conn, log);
    }
}
\end{lstlisting}
\caption{Метод из файла selected/180566.java}
\end{subfigure}
\end{figure}
\begin{figure}[H]\ContinuedFloat
\begin{subfigure}{\textwidth}
\begin{lstlisting}[style=java]
public void handleMessage(Message message) throws Fault {
    InputStream is = message.getContent(InputStream.class);
    if (is == null) {
        return;
    }
    CachedOutputStream bos = new CachedOutputStream();
    try {
        IOUtils.copy(is, bos);
        is.close();
        bos.close();
        sendMsg("Inbound Message \n" + "--------------" + bos.getOut().toString() + "\n--------------");
        message.setContent(InputStream.class, bos.getInputStream());
    } catch (IOException e) {
        throw new Fault(e);
    }
}
\end{lstlisting}
\caption{Метод из файла selected/1607050.java }
\end{subfigure}
\caption{Пример клона III типа из BigCloneBench}
\label{fig:third}
\end{figure}

Фильтрация данных была сделана по значению в столбце \texttt{syntactic\_type}. Выбирались только те значения, которые не превышали 3. Однако, не было учтено одинаковое значение в данном столбце у клонов III и IV типов. Вызвано это сложностью разделения клонов III и IV типов в выборке BigCloneBench. Клоны данных типов разделяются на основе их значений синтаксического сходства. Таким образом, выделяются три диапазона значения:

\begin{itemize}
\setlength\itemsep{0mm}
\item \([0.0, 0.5)\) - преимущественно клоны IV типа;
\item \([0.5, 0.7)\) - преимущественно клоны III типа;
\item \([0.7, 1.0)\) - строго III тип клонов.
\end{itemize}

\section{Описание тестовой платформы и конфигурации прототипа}

Все эксперименты в рамках данного тестирования были проведены на двух различных машинах со следующими конфигурациями:

\begin{enumerate}
\setlength\itemsep{0mm}
\item ПК
\begin{itemize}
\setlength\itemsep{0mm}
\item ОС ArchLinux;
\item CPU Intel(R) Core(TM) i7-7700K CPU 4.20ГГц;
\item 64 Гб RAM;
\item NVidia Quadro P4000 8Гб;
\end{itemize}
\item DGX-1
\begin{itemize}
\setlength\itemsep{0mm}
\item ОС Ubuntu 16.04 LTS
\item Intel(R) Xeon(R) CPU E5-2698 v4 2.20 ГГц x2;
\item 512 ГГб RAM
\item NVidia Tesla V100 16Гб x8.
\end{itemize}
\end{enumerate}

\section{Схема работы прототипа}

Общая схема работы прототипа в тестовом режиме выглядит следующим образом:

\begin{itemize}
\setlength\itemsep{0mm}
\item Случайным образом выбираются пары клонов и не клонов из таблиц BigCloneBench;
\item Производится анализ файлов соответствующих данной выборке;
\item Для проанализированных файлов строится PSI;
\item Полученные PSI преобразуются в последовательности интересующих нас токенов;
\item Для построенных токенов строятся их векторные преобразования с помощью word2vec;
\item С помощью seq2seq полученные преобразования приводятся к единой размерности;
\item Результат работы seq2seq передается в Сиамскую сеть, в которой производится сравнение пар;
\item На основе результата Сиамской сети считаются метрики для анализа работы прототипа.
\end{itemize}

\section{Исследование показателей прототипа}

Главной задачей данной работы является разработка метода интеллектуального обнаружения клонов. Основными требованиями, предъявляемыми к такому методу, являются увеличение точности поиска и определение клонов первых трех типов. Данные характеристики исследуются на примере использования разработанного прототипа для обнаружения клонов. Тестирование производится на сетях, обученных на двух разных выборках. В первом случае, Сиамская сеть была обучена на небольшой выборке из BigCloneBench. В качестве обучающей выборки использовался набор из 40000 пар: 30000 из них - клоны, остальные 10000 - нет. Результат анализа прототипа представлен в таблицах ниже (табл.~\ref{testing}~-~\ref{mutator}), где:

\begin{itemize}
\setlength\itemsep{0mm}
\item \(KLOC\) - количество строк исходного кода (в тысячах);
\item \(N_{methods}\) - количество исследуемых методов;
\item \(N_{clones}\) - количество найденных клонов;
\item \(Rec\) - полнота результата;
\item \(Prec\) - точность результата;
\item \(F_1\) - гармоническое среднее значение полноты и точности;
\item \(t_f\) - полное время работы (без учета обучения сетей).
\end{itemize}

\begin{table}[H]
\centering
\captionsetup{skip=5pt}
\caption{Результаты тестирования BigCloneBench}
\label{testing}
\begin{tabular}{|c|c|c|c|c|c|c|c|}
\hline
\(N_{methods}\) & \(KLOC\) & \(N_{clones}\) & \(Rec\)  & \(Prec\) & \(F_1\) & \(t_f\)        \\ \hline
\multicolumn{7}{|c|}{ПК}										   		    			   \\ \hline
4092			& 228	   & 1697			& 0.94	   & 0.97 	  & 0.95 	& 04:42 мин	   \\ \hline
8270			& 768  	   & 5453    		& 0.88 	   & 0.94 	  & 0.91 	& 06:55 мин	   \\ \hline
10892			& 3192 	   & 32041   		& 0.88 	   & 0.93 	  & 0.90 	& 09:44 мин 	   \\ \hline
19711    		& 2830 	   & 24797   		& 0.90 	   & 0.76 	  & 0.82 	& 14:43 мин 	   \\ \hline
\multicolumn{7}{|c|}{DGX}										   		      		   \\ \hline
4121			& 231	   & 2314    		& 0.92 	   & 0.95 	  & 0.93 	& 03:20 мин 	   \\ \hline
8321     		& 754  	   & 5325    		& 0.89 	   & 0.97 	  & 0.93 	& 04:34 мин 	   \\ \hline
10523    		& 3152 	   & 26395   		& 0.88 	   & 0.94 	  & 0.91 	& 07:55 мин 	   \\ \hline
18753    		& 2965 	   & 35486   		& 0.90 	   & 0.79 	  & 0.84 	& 11:32 мин 	   \\ \hline
\end{tabular}
\end{table}

Во втором случае обучение Сиамской сети производилось на результате работы <<мутатора>>. На вход разработанному <<мутатору>> подавался проект IntelliJ IDEA Community. Объем Java кода данного проекта - 3545 KLOC и содержит он в себе порядка 338 тыс. методов. Однако, так как обучение на одних клонах не является корректным, то необходимы были контрпримеры. В качестве контрпримера было решено взять другой проект - Incubator Netbeans. Объем Java кода - 3650 KLOC. Количество методов в данном проекте не превышает 283 тыс.

\begin{table}[H]
\centering
\captionsetup{skip=5pt}
\caption{Результаты тестирования <<мутатора>>}
\label{mutator}
\begin{tabular}{|c|c|c|c|c|c|c|}
\hline
\(N_{methods}\) & \(KLOC\) & \(N_{clones}\) & \(Rec\)  & \(Prec\) & \(F_1\) & \(t_f\)        \\ \hline
\multicolumn{7}{|c|}{ПК}                                       \\ \hline
3951       & 195  & 1348      & 0.73 & 0.76 & 0.74 & 06:31 мин \\ \hline
7954       & 679  & 4731      & 0.77 & 0.73 & 0.75 & 09:28 мин \\ \hline
12573      & 2803 & 24385     & 0.75 & 0.73 & 0.74 & 14:44 мин \\ \hline
20842      & 2947 & 32679     & 0.74 & 0.75 & 0.75 & 17:37 мин \\ \hline
\multicolumn{7}{|c|}{DGX}                                      \\ \hline
4334       & 206  & 1028      & 0.75 & 0.74 & 0.75 & 04:15 мин \\ \hline
8016       & 721  & 5267      & 0.72 & 0.76 & 0.74 & 06:22 мин \\ \hline
11845      & 1573 & 16612     & 0.73 & 0.77 & 0.75 & 12:47 мин \\ \hline
18539      & 3182 & 29761     & 0.76 & 0.78 & 0.74 & 15:14 мин \\ \hline
\end{tabular}
\end{table}

Все анализируемые метрики весьма тривиальные, за исключением \(Recall\), \(Precision\) и \(F_1\). Для их пояснения необходимо ввести некоторые понятия, а именно - \(True\ positives\ (TP)\), \(True\ negatives\ (TN)\), \(False\ positives\ (FP)\), \(False\ negatives\ (FN)\), где:

\begin{itemize}
\setlength\itemsep{0mm}
\item \(True\ positives\) - истинно-положительный результат;
\item \(True\ negatives\) -  истинно-отрицательный результат;
\item \(False\ positives\) - ложно-положительные результаты;
\item \(False\ negatives\) - ложно-отрицательные результаты.
\end{itemize}

Такие метрики рассчитываются согласно табл. \ref{truepos}, где CD - ответ выданный прототипом (Clone Detector).

\begin{table}[H]
\centering
\captionsetup{skip=5pt}
\caption{Вспомогательные метрики}
\label{truepos}
\begin{tabular}{c|c|c|}
\cline{2-3}
                                   & Клон & Не клон \\ \hline
\multicolumn{1}{|c|}{Клон (CD)}    & TP   & FP      \\ \hline
\multicolumn{1}{|c|}{Не клон (CD)} & FN   & TN      \\ \hline
\end{tabular}
\end{table}

Итак, после введения дополнительных метрик, рассчет \(Precision\), \(Recall\) и \(F_1\) будет намного нагляднее. Для расчета основных метрик оценки работы нейронных сетей используются следующие формулы:

\begin{equation}
\label{eq:recall}
Recall = \frac{True\ positives}{True\ positives + False\ negatives}
\end{equation}

\begin{equation}
\label{eq:precision}
Precision = \frac{True\ positives}{True\ positives + False\ positives}
\end{equation}

\begin{equation}
\label{eq:f1}
F_1 = 2*\frac{Precision * Recall}{Precision + Recall}
\end{equation}

\section{Результаты тестирования}

Главными характеристиками, которые анализировались при тестировании прототипа являются точность (\(Precision\)), полнота (\(Recall\)) и их гармоническое среднее значение \(F_1\). Однако, основное внимание необходимо уделить метрике \(F_1\), которая уравновешивает значения полноты и точности. Таким образом, чем больше значение (\(F_1\)) - тем лучше работает разработанный прототип.

Проанализировав результаты в таблице \ref{testing}, можно заметить следующее:
\begin{itemize}
\setlength\itemsep{0mm}
\item С увеличением размера проекта увеличивается время, затрачиваемое на анализ;
\item С увеличением размера проекта незначительно снижается качество работы прототипа.
\end{itemize}

Затрачиваемое время на анализ напрямую зависит от объема методов в анализируемой выборке. Связано это с последовательной обработкой методов и отсутствии вычислительной оптимизации.

<<Проседание>> точности вычисления связано с отсутствием четкого разграничения III и IV типов клонов. Таким образом, клоны IV типа распознаются неверно, что и является причиной падения точности.

Из результатов тестирования инструмента, обученного на <<мутированных>> данных следует следующее:
\begin{itemize}
\setlength\itemsep{0mm}
\item С увеличением размера проекта увеличивается время, затрачиваемое на анализ;
\item Точность анализа не превышает 75\%;
\item Точность анализа <<кучнее>> (по сравнению с результатами из табл.~\ref{testing}).
\end{itemize}

Увеличение времени объясняется теми же причинами, что и в предыдущем случае. Точность, в данном случае, напрямую зависит от обучающей выборке, которая содержит в себе множество неточностей и не обладает высоким качеством. <<Кучность>> обуславливается отсутствием размытых границ между клонами III и IV типов в данной выборке. В обучающем наборе клоны IV типа не рассматривались. Именно поэтому методы больше относящиеся к клонам IV типа, нежели к клонам III типа инструментом не определялись.

При рассмотрении скорости работы предлагаемого метода, можно сделать вывод о ее вариативности. На текущий момент, скорость выполнения скорее низкая, нежели средняя. Однако, данный инструмент, на текущий момент, не был оптимизирован для достижения максимальной скорости работы. 

\section{Итоги раздела}

В данном разделе было проведено тестирование разработанного прототипа, реализующего предлагаемый метод интеллектуального поиска клонов. Проанализировав результаты тестирования, можно сделать вывод о целесообразности использования данного подхода. Однако, тестирование так же показало, что для полноценного использования разработанного прототипа необходимо произвести его оптимизацию. 


