\chapter{Тестирование и анализ результатов}

В данном разделе проводится исследование характеристик прототипа на примере его использования для обнаружения клонов. Тестирование проводилось на нескольких различных выборках из BigCloneBench.

\section{Тестирование прототипа обнаружения клонов}

\subsection{Описание тестовых данных}

\subsection{Описание тестовой платформы и конфигурации прототипа}

Все эксперименты в рамках данного тестирования были проведены на двух различных машинах со следующими конфигурациями:

\begin{enumerate}
\setlength\itemsep{0mm}
\item ПК
\begin{itemize}
\setlength\itemsep{0mm}
\item ОС ArchLinux;
\item CPU Intel(R) Core(TM) i7-7700K CPU 4.20ГГц;
\item 64 Гб RAM;
\item NVidia Quadro P4000 8Гб;
\end{itemize}
\item DGX-1
\begin{itemize}
\setlength\itemsep{0mm}
\item ОС Ubuntu 16.04 LTS
\item Intel(R) Xeon(R) CPU E5-2698 v4 2.20 ГГц x2;
\item 512 ГГб RAM
\item NVidia Tesla V100 16Гб x8.
\end{itemize}
\end{enumerate}

\subsection{Исследование показателей прототипа}

Главной задачей данной работы является разработка метода интеллектуального обнаружения клонов. Основными требованиями, предъявляемыми к такому методу, являются увеличение точности поиска и определение клонов первых трех типов. Данные характеристики исследуются на примере использования разработанного прототипа для обнаружения клонов. Результат анализа прототипа представлен в таблице \ref{testing}, где:

\begin{itemize}
\setlength\itemsep{0mm}
\item \(KLOC\) - количество строк исходного кода (в тысячах);
\item \(N_{methods}\) - количество исследуемых методов;
\item \(N_{clones}\) - количество найденных клонов;
\item \(Rec\) - полнота результата;
\item \(Prec\) - точность результата;
\item \(t_w\) - время расчета (без учета преобразований и обучения).
\end{itemize}

\begin{table}[H]
\centering
\captionsetup{skip=5pt}
\caption{My caption}
\label{testing}
\begin{tabular}{|c|c|c|c|c|c|}
\hline
\(N_{methods}\) & KLOC  & \(N_{clones}\) & Rec  & Prec & \(t_w\)     \\ \hline
4178     & 286.5 & 1834      & 0.93 & 0.97 & 02:17 мин \\ \hline
         &       &           &      &      &       \\ \hline
         &       &           &      &      &       \\ \hline
\end{tabular}
\end{table}

Все анализируемые метрики весьма тривиальные, за исключением \(Recall\) и \(Precision\). Для их пояснения необходимо ввести некоторые понятия, а именно - \(True\ positives\ (TP)\), \(True\ negatives\ (TN)\),\(False\ positives\ (FP)\), \(False\ negatives\ (FN)\), где:

\begin{itemize}
\setlength\itemsep{0mm}
\item \(True\ positives\) - 
\item \(True\ negatives\) - 
\item \(False\ positives\) - ложно-положительные результаты;
\item \(False\ negatives\) - ложно-отрицательные результаты.
\end{itemize}

Такие метрики рассчитываются согласно табл. \ref{truepos}.

\begin{table}[H]
\centering
\captionsetup{skip=5pt}
\caption{My caption}
\label{truepos}
\begin{tabular}{c|c|c|}
\cline{2-3}
                                   & Клон & Не клон \\ \hline
\multicolumn{1}{|c|}{Клон (CD)}    & TP   & FP      \\ \hline
\multicolumn{1}{|c|}{Не клон (CD)} & FN   & TN      \\ \hline
\end{tabular}
\end{table}
