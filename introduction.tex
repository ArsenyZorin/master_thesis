%%%%%%%%%%%%%%%%%%%%%%%%%%%%%%%%%%%%%%%%%%%%%%%%%%%%%%%%%%%%%%%%%%%%%%%%%%%%%%%%
\intro
%%%%%%%%%%%%%%%%%%%%%%%%%%%%%%%%%%%%%%%%%%%%%%%%%%%%%%%%%%%%%%%%%%%%%%%%%%%%%%%%
Одной из актуальных проблем разработки и сопровождения программного обеспечения (ПО) является наличие клонов в исходном коде программы. Фрагмент кода, который был скопирован и вставлен с незначительными или большими изменениями называется дубликатом ПО или клоном. По данным различных исследований современное ПО насчитывает до 30\% клонов \cite{royandcordy}.
\nomenclature{ПО}{Программное обеспечение}

Клоны могут появляться в программных системах по разным причинам. Это может быть копирование участков когда. Такой подход может стать разумной отправной точкой для написания программы. Другая причина - разработка уже существующего участка кода другим программистом. Также причиной появления клонов может служить невыразительность используемого языка программирования или API (Application Program Interface - программный интерфейс приложения).

\nomenclature{API}{Application Program Interface}

Среди многих проблем, связанных с наличием клонов в исходном коде программы, одной из главных является ее последующее сопровождение. В том случае, когда имеется ошибка в одном из дублированных фрагментов, ее необходимо найти и справить во всех идентичных участках, а не только в одном.

Обнаружение клонов является важной проблемой для поддержки ПО. Однако, несмотря на тот факт, что в предыдущих исследованиях были показаны отрицательное воздействие клонировния, это не всегда так. Постоянная необходимость изменения клонов также отсутствует. Тем не менее, возможность обнаружения двух идентичных участков кода критична во многих областях, например, упрощение и совместимость кода, определение плагиата, нарушение авторских прав, обнаружение вредоносного ПО и т.д.

Целью данной работы является разработка метода поиска клонов основанного на использовании нейронных сетей. На основе данного метода планируется разработка и тестирование прототипа инструмента поиска клонов.

Работа состоит из 5 разделов. В первом разделе представлен обзор и сравнительный анализ существующих подходов, введены основные определения и классификации, описаны причины возникновения клонов. Во втором разделе сформулированы требования к предлагаемому подходу и прототипу инструмента. Третий раздел посвящен разработке интеллектуального метода обнаружения клонов, а также используемым алгоритмам в ходе разработки метода. В четвертом разделе описана реализация предлагаемого подхода и прототипа инструмента для анализа Java-программ. Тестированию инструмента обнаружения клонов в исходном коде программ посвящен пятый раздел.
