
\keywords{%
  машинное обучение,
  искусственные нейронные сети,
  поиск клонов
}

\abstractcontent{
Магистерская работа посвящена исследованию возможности применения искусственных нейронных сетей в задаче поиска клонов в исходном коде программ. Исследование было проведено на основе проектов с открытым исходным кодом, которые написаны на языке программирования Java. В ходе проведения данного исследования был разработан прототип инструмента для поиска дубликатов в исходном коде программы.

Разработанный прототип был протестирован на специальной выборке - BigCloneBench. Тестирование показало целесообразность использования метода интеллектуального поиска клонов.
}

\keywordsen{
  code clone detection,
  machine learning,
  neural networks
}

\abstractcontenten{
This work is dedicated to analysis of artificial neural network usage for clone detection in source code of a program. The AI-based approach was proposed. The work is based on Recurrent Neural Networks usage. In the course of the study, a prototype of a tool for clone detection in a source code was developed.

The developed prototype was tested on a special BigCloneBench benchmark. Results of conducted experiments show that proposed technique is expedient for usage in clone detection tasks.


}
