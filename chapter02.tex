%%%%%%%%%%%%%%%%%%%%%%%%%%%%%%%%%%%%%%%%%%%%%%%%%%%%%%%%%%%%%%%%%%%%%%%%%%%%%%%%
\chapter{Постановка задачи и выбор пути решения}
%%%%%%%%%%%%%%%%%%%%%%%%%%%%%%%%%%%%%%%%%%%%%%%%%%%%%%%%%%%%%%%%%%%%%%%%%%%%%%%%

В данном разделе рассматриваются основные требования, предъявляемые к методу обнаружения клонов. Приводится постановка и подробное описание задач, решаемых в рамках данной работы, в том числе:
\begin{itemize}
\setlength\itemsep{0mm}
\item задача разработки метода обнаружения клонов, основанного на использовании искусственных нейронных сетей
\item задача разработки прототипа инструмента обнаружения
\end{itemize}
%%%%%%%%%%%%%%%%%%%%%%%%%%%%%%%%%%%%%%%%%%%%%%%%%%%%%%%%%%%%%%%%%%%%%%%%%%%%%%%%
\section{Задача разработки интеллектуального метода обнаружения клонов}
%%%%%%%%%%%%%%%%%%%%%%%%%%%%%%%%%%%%%%%%%%%%%%%%%%%%%%%%%%%%%%%%%%%%%%%%%%%%%%%%

Целью данной работы является разработка метода обнаружения клонов. Главное требование, которое предъявляется к разрабатываемому методу - использование искусственных нейронных сетей. Кроме того, разрабатываемый метод интеллектуального обнаружения клонов должен обеспечивать решение следующих задач:
\begin{itemize}
\setlength\itemsep{0mm}
\item обнаружение клонов I-III типов;
\item максимизация полноты и точности обнаружения;
\item извлечение информации о клонах в виде клоновых классов.
\end{itemize}



%%%%%%%%%%%%%%%%%%%%%%%%%%%%%%%%%%%%%%%%%%%%%%%%%%%%%%%%%%%%%%%%%%%%%%%%%%%%%%%%
\section{0xC0DECAFE}
%%%%%%%%%%%%%%%%%%%%%%%%%%%%%%%%%%%%%%%%%%%%%%%%%%%%%%%%%%%%%%%%%%%%%%%%%%%%%%%%

You can review what you have to do in figure~\ref{fig:how-to-do-research}.
Please do so now. \blindtext

\Blindtext
