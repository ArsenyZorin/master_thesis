%%%%%%%%%%%%%%%%%%%%%%%%%%%%%%%%%%%%%%%%%%%%%%%%%%%%%%%%%%%%%%%%%%%%%%%%%%%%%%%%
\chapter{Постановка задачи разработки интеллектуального метода обнаружения клонов и выбор пути решения}
%%%%%%%%%%%%%%%%%%%%%%%%%%%%%%%%%%%%%%%%%%%%%%%%%%%%%%%%%%%%%%%%%%%%%%%%%%%%%%%%

Задача данного исследования заключается в разработке интеллектуального метода обнаружения программных клонов. Основная идея предлагаемого подхода заключается в использовании искусственных нейронных сетей на стадии поиска клонов в исходном коде. Таким образом, предлагаемый интеллектуальный метод получит возможность обнаруживать клоны первых трех типов.

В данном разделе рассматриваются основные требования, предъявляемые к методу обнаружения клонов. Приводится постановка и подробное описание задач, решаемых в рамках данной работы, в том числе:
\begin{itemize}
\setlength\itemsep{0mm}
\item задача разработки метода обнаружения клонов, основанного на использовании искусственных нейронных сетей
\item задача разработки прототипа инструмента обнаружения
\end{itemize}
%%%%%%%%%%%%%%%%%%%%%%%%%%%%%%%%%%%%%%%%%%%%%%%%%%%%%%%%%%%%%%%%%%%%%%%%%%%%%%%%
\section{Задача разработки интеллектуального метода обнаружения клонов}
%%%%%%%%%%%%%%%%%%%%%%%%%%%%%%%%%%%%%%%%%%%%%%%%%%%%%%%%%%%%%%%%%%%%%%%%%%%%%%%%

Целью данной работы является разработка метода для обнаружения клонов. Главное требование, которое предъявляется к разрабатываемому методу - обнаружение программных клонов первых трех типов. Нейронные сети, в данной задаче, используются из-за того, что они очень хорошо подходят именно под задачи поиска схожих элементов. Таким образом, разрабатываемый метод интеллектуального обнаружения клонов должен обеспечивать решение следующих задач:
\begin{itemize}
\setlength\itemsep{0mm}
\item обнаружение клонов I-III типов;
\item максимизация полноты и точности обнаружения;
\item извлечение информации о клонах в виде клоновых классов.
\end{itemize}

Несмотря на тот факт, что разрабатываемый метод должен быть интеллектуальным, в качестве предварительной обработки необходимо привести исходный код к одному из подходящих видов внутреннего представления. В предлагаемом подходе было решено использовать несколько представлений (гибридный метод). Этими представлениями являются AST и последовательность токенов. 

Использование AST позволяет сохранить информацию о структуре исходного текста программы, именно благодаря этому становится возможным увеличить полноту и точность результатов. Использование токенов в качестве представления исходного кода позволяет, в свою очередь, значительно сократить размер входных данных. Общую структуру внутреннего представления кода можно увидеть на рис.~\ref{fig:stages}

\begin{figure}[htbp]
\centering
\includegraphics[width=\textwidth]{stages.png}
\caption{Структура внутреннего представления кода}
\label{fig:stages}
\end{figure}

\section{Искусственные нейронные сети}

На текущий момент искусственные нейронные сети (НС) достигли высокой производительности в таких задачах поиска схожих элементов, как, например, поиск одинаковых изображений, фотографий и текста. Данное достижение - одна из основных причин, по которой было решено использовать искусственные нейронные сети.

\nomenclature{НС}{Нейронная сеть}

Для того, чтобы определить, какая из архитектур НС больше всего подходит для решения поставленной задачи, необходимо провести сравнительный анализ. 

\subsection{Сеть прямого распространения}

Как следует из названия, такая сеть передает информацию только в одном направлении, от входа к выходу. НС состоят из полносвязных слоев (каждый нейрон из одного слоя связан с каждым нейроном следующего слоя), однако, сам слой между собой никак не связан. Каждый слой состоит из входных, скрытых или выходных ячеек~\cite{perceptron}. 

Круг применения таких сетей весьма узок, их можно применять, например, для простых задач классификации или предсказаний. 

\subsection{Нейронная сеть Хопфилда}

Рассматриваемая сеть является полносвязной НС с симметричной матрицей связей. Каждый узел, в такой сети, является входным до начала процесса обучения, скрытым - во время и выходным после обучения. Обучение сети производится с помощью установления желаемого значения нейрона, после чего могут быть рассчитаны веса~\cite{hopfield}. 

Как только веса заданы, обученная сеть становится способной <<распознавать>> входные сигналы - то есть, определять, к какому из запомненных образцов они относятся.

\subsection{Сверточная нейронная сеть}

Сверточная НС отличается от остальных. Основной ее задачей является обработка изображений. Довольно типичной задачей для нее является классификация или обнаружение объектов на изображениях. Как правило, такие сети начинают свою работу с, так называемого, входного ''сканера'', который не пытается анализировать все входные данные разом, а анализирует их небольшими фрагментами, соответствующими его размерам~\cite{cnn}. 

\subsection{Рекуррентная нейронная сеть}

В данном виде НС связи между нейронами образуют своеобразный направленный граф. Благодаря такому строению появляется возможность обработки серии событий во времени или последовательные цепочки. Однако узким местом таких сетей является проблема исчезающего градиента, то есть информация со временем теряется~\cite{rnn}.

\subsection{Сравнительный анализ}

Основываясь на проведенном анализе архитектур искусственных нейронных сетей, можно заключить, что такие сети как сети прямого распространения или сверточные нейронные сети абсолютно не подходят под задачу анализа исходного кода программ.
\section{Задача разработки прототипа инструмента}

Одна из задач данной работы заключается в разработке и тестировании прототипа на базе предложенного метода. Такой инструмент должен обладать следующими свойствами:
\begin{itemize}
\setlength\itemsep{0mm}
\item Создание внутреннего представления кода;
\item Использование рекуррентных нейронных сетей для поиска клонов;
\item Поиск клонов I-III типов;
\item Достижение точности и полноты поиска не ниже 80\% на тестовой выборке, основанной на BigCloneBench~\cite{bcb}.
\end{itemize}

Для данного инструмента, в качестве целевого языка программирования был выбран язык Python. Такой выбор был сделан из-за популярности и частым использованием его в области разработки нейронных сетей. К тому же, для данного языка программирования существует множество библиотек для реализации и обучения нейронных сетей. Одна из них - Tensorflow, которая и была использована в данной работе~\cite{tf}.

\section{Итоги раздела}

В данном разделе поставлена задача разработки интеллектуального метода обнаружения программных клонов. Также поставлена задача разработки прототипа инструмента, работающего на основе предложенного метода и выполняющего поиск дублированных фрагментов в исходном коде программ. Были выделены основные свойства, которыми должен обладать разрабатываемый прототип.
% Учитывая выбранный подход, задача реализации прототипа состоит из двух частей:
% 
% \begin{enumerate}
% \item Разработка инструмента, производящего преобразование исходного кода в выбранное представление;
% \item Разработка инструмента, использующего нейронные сети для поиска клонов.
% \end{enumerate}
% 
% \subsection{Разработка инструмента для создания внутреннего представления}

% К разрабатываемому инструменту предъявляются следующие требования:
% \begin{itemize}
% \setlength\itemsep{0mm}
% \item Получение исходного кода на языка Java из заданной директории;
% \item Построение AST для полученного кода;
% \item Фильтрация AST;
% \item Преобразование AST в последовательность токенов.
% \end{itemize}
% 
% В качестве целевого языка был выбран язык Java, так как на протяжении многих лет он является одним из самых распространенных и используемых языков программирования (согласно TIOBE)~\cite{TIOBE}. 
% 
% \subsection{Разработка инструмента для поиска программных клонов}
% 
% Вторая важная часть разработки прототипа - реализация инструмента поиска клонов. К такому инструменту предъявляются следующие требования:
% \begin{itemize}
% \setlength\itemsep{0mm}
% \item Использование рекуррентных нейронных сетей для поиска клонов;
% \item Поиск клонов I-III типов;
% %\item высокая точность и полнота поиска
% \item Точность и полнота поиска не ниже 80\% на тестовой выборке, основанной на BigCloneBench.
% \end{itemize}
% 
% Для данного инструмента, в качестве целевого языка программирования был выбран язык Python. Такой выбор был сделан из-за популярности и частым использованием его в области разработки нейронных сетей. К тому же, для данного языка программирования существует множество библиотек для реализации и обучения нейронных сетей. Одна из них - Tensorflow, которая и была использована в данной работе~\cite{tf}.
